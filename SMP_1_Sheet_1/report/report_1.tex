\RequirePackage[l2tabu,orthodox]{nag} % turn on warnings because of bad style
\documentclass[a4paper,11pt,bibtotoc]{scrartcl}

\usepackage[utf8]{inputenc}     % This allows to type UTF-8 characters like ä,ö,ü,ß

\usepackage[T1]{fontenc}        % Tries to use Postscript Type 1 Fonts for better rendering
\usepackage{lmodern}            % Provides the Latin Modern Font which offers more glyphs than the default Computer Modern
\usepackage[intlimits]{amsmath} % Provides all mathematical commands

\usepackage{hyperref}           % Provides clickable links in the PDF-document for \ref
\usepackage{grffile}            % Allow you to include images (like graphicx). Usage: \includegraphics{path/to/file}

% Allows to set units
\usepackage{siunitx}
\sisetup{per-mode=fraction}     % Optional

% Additional packages
\usepackage{url}                % Lets you typeset urls. Usage: \url{http://...}
\usepackage{breakurl}           % Enables linebreaks for urls
\usepackage{xspace}             % Use \xpsace in macros to automatically insert space based on context. Usage: \newcommand{\es}{ESPResSo\xspace}
\usepackage{xcolor}             % Obviously colors. Usage: \color{red} Red text
\usepackage{booktabs}           % Nice rules for tables. Usage \begin{tabular}\toprule ... \midrule ... \bottomrule

% Source code listings
\usepackage{listings}           % Source Code Listings. Usage: \begin{lstlisting}...\end{lstlisting}

\definecolor{codegreen}{rgb}{0,0.6,0}
\definecolor{codegray}{rgb}{0.5,0.5,0.5}
\definecolor{codepurple}{rgb}{0.58,0,0.82}
\definecolor{backcolour}{rgb}{0.95,0.95,0.92}

\lstdefinestyle{mystyle}{
	backgroundcolor=\color{backcolour},   
	commentstyle=\color{codegreen},
	keywordstyle=\color{magenta},
	numberstyle=\tiny\color{codegray},
	stringstyle=\color{codepurple},
	basicstyle=\ttfamily\footnotesize,
	breakatwhitespace=false,         
	breaklines=true,                 
	captionpos=b,                    
	keepspaces=true,                 
	numbers=left,                    
	numbersep=5pt,                  
	showspaces=false,                
	showstringspaces=false,
	showtabs=false,                  
	tabsize=2
}

\lstset{style=mystyle}

\begin{document}

\titlehead{Simulation Methods in Physics I \hfill WS 2021/2022}
\title{Template for a nice \LaTeX{} document}
\author{NAME1, NAME2}
\date{\today}
\publishers{Institute for Computational Physics, University of
  Stuttgart}
\maketitle

\tableofcontents

\section{Math}

The package \verb|amsmath| is available for typesetting math. Execute \verb|texdoc amsmath| in your terminal for additional information.

Here is an equation typeset using commands from \verb|amsmath|.
%
\begin{align}
	\alpha &= 
	\begin{pmatrix}
		D_1t       & -a_{12}t_2 & \dots & -a_{1n}t_n \\
		-a_{21}t_1 & D_2t       & \dots & -a_{2n}t_n \\
		\hdotsfor[2]{4} \\
		-a_{n1}t_1 & -a_{n2}t_2 & \dots & D_nt
	\end{pmatrix} \\
	\beta &= \pi
\end{align}
%
Furthermore don't use \verb|eqnarray| and use \verb|align| instead! (If you don't know what \verb|eqnarray| is, nevermind)

\subsection{Units}

Typesetting units by hand is extremely annoying because you have to take care of spacing and font style, for example
%
\begin{verbatim}
100 \; \mathrm{m} \text{ or } 100 \; \mathrm{\frac{m}{s}}
\end{verbatim}
%
to produce \[ 100 \; \mathrm{m} \text{ or } 100 \; \mathrm{\frac{m}{s}} \]
That's why we use the package \verb|siunitx| See:
%
\begin{verbatim}
\SI{100}{\meter} \text{ or } \SI{100}{\meter\per\second}
\end{verbatim}
%
to produce $\SI{100}{\meter} \text{ or } \SI{100}{\meter\per\second}$

\section{Code}
You can include code using \verb|lstlisting|

\begin{lstlisting}[language=Python]
class HarmonicPotential():
	def __init__(self, k):
		self.k = k
	def calc_energy(self, x):
		return 0.5*self.k*x**2
	def calc_force(self, x):
		return -self.k*x
		
\end{lstlisting}

\end{document}

