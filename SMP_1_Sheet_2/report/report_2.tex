\RequirePackage[l2tabu,orthodox]{nag} % turn on warnings because of bad style
\documentclass[a4paper,11pt,bibtotoc]{scrartcl}

\usepackage[utf8]{inputenc}     % This allows to type UTF-8 characters like ä,ö,ü,ß

\usepackage[T1]{fontenc}        % Tries to use Postscript Type 1 Fonts for better rendering
\usepackage{lmodern}            % Provides the Latin Modern Font which offers more glyphs than the default Computer Modern
\usepackage[intlimits]{amsmath} % Provides all mathematical commands

\usepackage{hyperref}           % Provides clickable links in the PDF-document for \ref
\usepackage{grffile}            % Allow you to include images (like graphicx). Usage: \includegraphics{path/to/file}

% Allows to set units
\usepackage{siunitx}
\sisetup{per-mode=fraction}     % Optional

% Additional packages
\usepackage{url}                % Lets you typeset urls. Usage: \url{http://...}
\usepackage{breakurl}           % Enables linebreaks for urls
\usepackage{xspace}             % Use \xpsace in macros to automatically insert space based on context. Usage: \newcommand{\es}{ESPResSo\xspace}
\usepackage{xcolor}             % Obviously colors. Usage: \color{red} Red text
\usepackage{booktabs}           % Nice rules for tables. Usage \begin{tabular}\toprule ... \midrule ... \bottomrule

% Source code listings
\usepackage{listings}           % Source Code Listings. Usage: \begin{lstlisting}...\end{lstlisting}

\definecolor{codegreen}{rgb}{0,0.6,0}
\definecolor{codegray}{rgb}{0.5,0.5,0.5}
\definecolor{codepurple}{rgb}{0.58,0,0.82}
\definecolor{backcolour}{rgb}{0.95,0.95,0.92}

\lstdefinestyle{mystyle}{
	backgroundcolor=\color{backcolour},   
	commentstyle=\color{codegreen},
	keywordstyle=\color{magenta},
	numberstyle=\tiny\color{codegray},
	stringstyle=\color{codepurple},
	basicstyle=\ttfamily\footnotesize,
	breakatwhitespace=false,         
	breaklines=true,                 
	captionpos=b,                    
	keepspaces=true,                 
	numbers=left,                    
	numbersep=5pt,                  
	showspaces=false,                
	showstringspaces=false,
	showtabs=false,                  
	tabsize=2
}

\lstset{style=mystyle}

\begin{document}

\titlehead{Simulation Methods in Physics I \hfill WS 2021/2022}
\title{Worksheet 2: Statistical Mechanics and Molecular Dynamics}
\author{Lena Weber, Niklas Abraham, Cedric Förch}
\date{\today}
\publishers{Institute for Computational Physics, University of
  Stuttgart}
\maketitle

\tableofcontents

\section{Exercise 2.1}

We know from our lecture that the entropy of a system is related to the number of microstates $\Omega$ using the Boltzmann entropy formula:
\begin{align}
	S = k_B \ln(\Omega)
\end{align}

In the first case of N distinguishable particles, each of which can occupy one of two energy states $\epsilon_1$ and $\epsilon_2$, and a system in equilibrium with $n_1$ particles in state $\epsilon_1$ and $n_2$ particles in state $\epsilon_2$ such that $n_1 + n_2 = N$, we can express the entropy in terms of $\Omega$:
In order to solve this task, we use binomial coefficients to calculate the number of microstates $\Omega$:
\begin{align}
	\Omega = \binom{N}{n_1} = \frac{N!}{n_1! \cdot (N - n_1)!} \\
\end{align}




\end{document}

